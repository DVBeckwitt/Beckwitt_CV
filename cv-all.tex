% Consolidated CV file combining all content and package settings
% Generated to provide a single .tex source
\begin{filecontents*}{own-bib.bib}
@article{arendse2023insights,
  author  = {Christopher J. Arendse and Randy Burns and David Beckwitt and Dallar Babaian and Ping Yu and Suchismita Guha and Daniel Hill and Zahid Hossain and Paul Miceli},
  title   = {{Insights into the Growth Orientation and Phase Stability of {Chemical‑Vapor‑Deposited} {Two‑Dimensional} Hybrid Halide Perovskite Films}},
  journal = {ACS Applied Materials \& Interfaces},
  volume  = {15},
  number  = {50},
  pages   = {56692--56703},
  year    = {2023},
  month   = {Dec},
  doi     = {10.1021/acsami.3c14559},
  url     = {https://doi.org/10.1021/acsami.3c14559},
}

@inproceedings{beckwitt2020research,
  author      = {David Beckwitt},
  title       = {{Fabrication and Characterization of {2D} Heterostructure of {Graphene} and {Transition‑Metal Oxides}}},
  year        = {2020},
  abstract    = {The fabrication and characterization of two-dimensional (2D) heterostructures composed of graphene and layered transition‑metal oxides were investigated to elucidate their structural and optoelectronic properties for advanced material applications. Pulsed laser deposition (PLD) was employed to synthesize the heterostructures, enabling precise control over composition and thickness. The resulting materials were systematically characterized using X‑ray diffraction to assess crystallographic structure, Raman spectroscopy to probe vibrational modes and electronic interactions, and scanning electron microscopy to examine surface morphology and microstructural features. The study reveals critical insights into the interfacial phenomena and property enhancements arising from the integration of these materials, highlighting their potential for applications in next‑generation electronic, photonic, and energy technologies.},
  institution  = {Missouri State University},
  address      = {Missouri State University},
}

@inproceedings{beckwitt:inbre:2018,
  author    = {David Beckwitt},
  title     = {{Investigation of Solid‑State {LiPON} Thin Films Grown by {Pulsed Laser Deposition} for Application as an Electrolyte}},
  booktitle = {Arkansas {INBRE} Poster Presentation},
  year      = {2018},
  address   = {University of Arkansas},
  month     = {Oct},
}

@inproceedings{beckwitt:einhellig:2017,
  author    = {David Beckwitt},
  title     = {{Synthesis of {PbO}$_2$ Thin Films for Perovskite {CH}$_3${PbX}$_3$‑Based Solar Cell}},
  booktitle = {Einhellig Interdisciplinary Forum},
  year      = {2017},
  address   = {Springfield, MO},
}

@inproceedings{beckwitt:einhellig:2018,
  author    = {David Beckwitt},
  title     = {{Development of a Sol‑Gel {TiO}$_2$ Buffer Layer for Perovskite Solar Cell Applications}},
  booktitle = {Einhellig Interdisciplinary Forum},
  year      = {2018},
  address   = {Springfield, MO},
}

@inproceedings{beckwitt:talks:2023-24,
  author    = {David Beckwitt},
  title     = {{X‑Ray Diffraction Investigation of Disorder in {Van der Waals} Thin Films}},
  booktitle = {APS March Meeting, {ACNS}, and {MSU} Seminar (3 Talks)},
  year      = {2024},
  note      = {Presented at APS Prairie Section 2023 and at multiple venues in 2024},
}

@unpublished{beckwitt2025_giwaxs,
  author   = {Beckwitt, David and others},
  title    = {{Quantitative Modeling of Grazing‑Incidence Wide‑Angle X‑Ray Scattering Patterns from {Van der Waals} Thin Films}},
  year     = {2025},
  note     = {In Review—Anticipated Summer 2025},
  keywords = {inprep},
}

@unpublished{beckwitt2025_stackfaults,
  author   = {Beckwitt, David and others},
  title    = {{Quantitative Simulation of Stacking Faults and Structural Disorder in {CVD}‑Grown {PbI}$_2$ Thin Films}},
  year     = {2025},
  note     = {In Review—Anticipated Fall 2025},
  keywords = {inprep},
}

@unpublished{beckwitt2026_polytypism,
  author   = {Beckwitt, David and others},
  title    = {{Simulation‑Guided Control of Polytypism in {CVD}‑Grown {PbI}$_2$}},
  year     = {2026},
  note     = {In Preparation—Anticipated Spring 2026},
  keywords = {inprep},
}

@unpublished{beckwitt2026_cnn,
  author   = {Beckwitt, David and others},
  title    = {{Supervised Convolutional Neural Networks Trained on Simulated {GIWAXS} Patterns for Structural Analysis of Thin Films}},
  year     = {2026},
  note     = {In Preparation—Anticipated Spring 2026},
  keywords = {inprep},
}
\end{filecontents*}
% Main CV document with all packages and sections embedded
\documentclass[a4paper,skipsamekey,11pt,english]{curve}
\PassOptionsToPackage{style=ieee,sorting=ydnt,uniquename=init,defernumbers=true}{biblatex}
\makeatletter
%%%%%%%%%%%%%%%

\RequirePackage{silence}
\WarningsOff[longtable]
\WarningsOff[array]
\RequirePackage{array}

\usepackage{ifxetex,ifluatex}
\newif\ifxetexorluatex
\ifxetex
  \xetexorluatextrue
\else
  \ifluatex
    \xetexorluatextrue
  \else
    \xetexorluatexfalse
  \fi
\fi

\RequirePackage{graphicx}
\RequirePackage[hyphens]{url}
\RequirePackage{babel}
\raggedright

\RequirePackage[fixed]{fontawesome5}
\RequirePackage{simpleicons}

\newcommand{\smallcaps}[1]{\textsc{\lowercase{#1}}}
\usepackage{hyperref}   % makes images/ text clickable

\RequirePackage[a4paper,nohead,nofoot,left=2.75cm,right=2.25cm,vmargin=2cm]{geometry}
\RequirePackage{relsize}
\RequirePackage[dvipsnames,svgnames]{xcolor}
\RequirePackage{tikz}
\usetikzlibrary{shapes,shadows}

\RequirePackage{comment}
\RequirePackage{enumitem}
% Consistent bullet alignment across rubric sections
% and a uniform key column width for all rubrics
\newlength{\CVkeywidth}
\AtBeginDocument{%
  % set width based on the longest key/date used in the CV
  \settowidth{\CVkeywidth}{May 2026 (expected)}%
  % ensure all rubric tables share this width for the key column
  % override column type used for the date/key field so every rubric
  % reserves the same width; this keeps itemize bullets aligned across
  % different rubrics
  \newcolumntype{k}{>{\@keyfont}p{\CVkeywidth}}%
  % consistent bullet indentation for itemize environments
  \setlist[itemize]{align=parleft,leftmargin=2em,labelsep=0.5em,itemsep=0pt}
}
\definecolor{SwishLineColour}{HTML}{4A90E2}
\definecolor{MarkerColour}{HTML}{F5A623}
\newcommand{\RMPbadge}{%
  \href{https://www.ratemyprofessors.com/professor/2686648}{%
    {\small\color{MarkerColour!80!black}\textsc{Student Reviews}}%
  }%
}
% If you're not a researcher nor an academic, you probably don't need biblatex; delete this line.
\RequirePackage{biblatex}
\RequirePackage{csquotes}
\RequirePackage{tikz}
\newcommand*\circled[1]{\tikz[baseline=(char.base)]{
   \node[shape=circle,text=white,fill=MarkerColour!80!black,font=\sffamily\scriptsize\bfseries,inner sep=1pt,text height=1.35ex,minimum width=1.5em,text centered] (char) {#1};}}
% Simple bullet icon used for bibliography labels
\newcommand* \bibbullet{\tikz[baseline=(char.base)]{
  \node[shape=circle,fill=MarkerColour!80!black,inner sep=.5pt,
        text height=1ex,minimum width=1em] (char) {};
}}

% !TEX encoding = UTF-8
%\renewcommand{\entry}[2][]{\noindent\textbf{#1} #2\\}
\newcounter{bibitem}
\AtBeginBibliography{\setcounter{bibitem}{1}}
\iffieldformatundef{labelnumberwidth}{%
  % if author-year style
  \AtEveryBibitem{\makebox[2.5em][l]{\bibbullet}\stepcounter{bibitem}}%
}{%
  % if numeric style
  \DeclareFieldFormat{labelnumberwidth}{\makebox[2.5em][l]{\bibbullet}}%
  \setlength{\biblabelsep}{0pt}%
  \AtEveryBibitem{\stepcounter{bibitem}}%
}
\AtEveryBibitem{\iffieldundef{doi}{}{\clearfield{url}}}

% \renewcommand{\bibfont}{\small}
\setlength{\bibitemsep}{1.5ex}
\setlength{\bibhang}{2.5em}
\RequirePackage{xpatch}
\xpretofieldformat{doi}
  {\textcolor{MarkerColour!80!black}{\scriptsize\faLink}}
  {}{}
\xpretofieldformat{url}
  {\textcolor{MarkerColour!80!black}{\scriptsize\faLink}}
  {}{}


\usepackage{xcolor}

% Add this to your preamble
\definecolor{AccentBlue}{HTML}{1F4E79}

% Revised tagline macro for better spacing and readability
\newcommand{\tagline}{%
  \vspace{4pt} % increased vertical space from top contact details
  {\large\sffamily\bfseries\color{AccentBlue}%
   Computational Materials Science • Machine Learning • X-ray/Neutron Diffraction %
  }\par\vspace{4pt}
  {\small\sffamily\color{gray}%
   Quantitative Diffraction Simulation• 2D Materials • Defect Engineering • Thin-Film Growth%
  }%
  \vspace{8pt} % additional bottom spacing
}

\headerscale{1}
%\setlength{\headerspace}{6pt}
\rubricfont{\Large\bfseries\sffamily}
\setlength{\rubricspace}{0.3cm}
%\setlength{\rubricafterspace}{-9pt}
\setlength{\rubricafterspace}{-3pt}
\setlength{\subrubricspace}{3pt}
\setlength{\subrubricbeforespace}{4pt}
\def\@@rubrichead#1{%
  \begin{tikzpicture}[baseline]%\
  \shade[left color=SwishLineColour!60!white, right color=white] rectangle (\@almosttextwidth,2.5pt);
  \node[font={\@rubricfont},inner sep=0pt,text ragged,anchor=south west,text depth=.5ex,text height=1.5ex] at (1pt,2pt) {#1};
  \end{tikzpicture}%
  \vspace\rubricspace%
}

\subrubricfont{\large\bfseries\sffamily}
\subrubricalignment{l}

\newcommand{\makefield}[2]{\makebox[1.5em]{\color{MarkerColour!80!black}#1} #2\hspace{2em}}

\keyalignment{r}
\rubricalignment{l}
\renewcommand{\arraystretch}{1.25}
\urlstyle{tt}

\newcommand{\prefixmarker}[1]{\def\@prefixmarker{#1}}
\def\@prefixmarker{\relscale{.9}\faBookmark}

% Align prefix markers and entry text regardless of marker width
\prefix{%
  \makebox[1.5em][c]{\color{MarkerColour!80!black}\@prefixmarker}%
  \hspace*{0.5em}%
}

\newcommand{\makerubrichead}[1]{\vskip\baselineskip\@@rubrichead{#1}}

\defbibheading{subbibliography}{\vskip\subrubricbeforespace{\@subrubricfont\hspace{3pt}#1}\par}

\defbibfilter{booksandchapters}{%
( type=book or type=incollection )
}

\RequirePackage{pgffor}
\newcommand{\mynames}[1]{\def\my@namelist{#1}}
\newtoggle{boldname}
\renewcommand*{\mkbibnamefamily}[1]{%
  \global\togglefalse{boldname}%
  \foreach \my@fname / \my@gname in \my@namelist {%
    \ifboolexpr{ test {\ifdefstrequal{\namepartfamily}{\my@fname}}
                 and
                 test {\ifdefstrequal{\namepartgiven}{\my@gname}}}
      {\global\toggletrue{boldname}}{}%
  }%
  \iftoggle{boldname}{\textbf{#1}}{#1}%
}

\renewcommand*{\mkbibnamegiven}[1]{%
  \global\togglefalse{boldname}%
  \foreach \my@fname / \my@gname in \my@namelist{%
    \ifboolexpr{ test {\ifdefstrequal{\namepartfamily}{\my@fname}}
                 and
                 test {\ifdefstrequal{\namepartgiven}{\my@gname}}}
      {\global\toggletrue{boldname}\breakforeach}{}%
  }%
  \iftoggle{boldname}{\textbf{#1}}{#1}%
}


\RequirePackage[colorlinks=true,allcolors=gray!40!black,breaklinks=true]{hyperref}


\makeatother
\usepackage{multicol}
\DefineBibliographyStrings{english}{url={\textsc{url}}}
\addbibresource{own-bib.bib}
\mynames{Beckwitt/David}
%% MAKE SURE THERE IS NO SPACE AFTER THE FINAL NAME IN YOUR \mynames LIST
\ifxetexorluatex
  \usepackage{fontspec}
  \usepackage[p,osf,swashQ]{cochineal}
  \usepackage[medium,bold]{cabin}
  \usepackage[varqu,varl,scale=0.9]{zi4}
\else
  \usepackage[T1]{fontenc}
  \usepackage[p,osf,swashQ]{cochineal}
  \usepackage{cabin}
  \usepackage[varqu,varl,scale=0.9]{zi4}
\fi
\includecomment{fullonly}
\leftheader{
  {\LARGE\bfseries\sffamily David Beckwitt, Ph.D.}
  \makefield{\faEnvelope[regular]}{\href{mailto:David.Beckwitt@gmail.com}{\texttt{David.Beckwitt@gmail.com}}}
  \makefield{\faGithub}{\href{https://github.com/DVBeckwitt}{\texttt{@DVBeckwitt}}}
  \makefield{\faLinkedin}{\href{https://www.linkedin.com/in/DVBeckwitt/}{\texttt{@DVBeckwitt}}}
  \makefield{\faResearchgate}{\href{https://www.researchgate.net/profile/David-Beckwitt-3}{\texttt{DVBeckwitt}}}
  \\
  { Ph.D. Candidate Physicist specializing in quantitative modeling and machine learning analysis of structural defects in van der Waals materials.}
}
\rightheader{~}
\title{Curriculum Vitae}
\begin{document}
\makeheaders[c]
%{\color{TaglineGrey}\tagline}
% Academic background
\begin{rubric}{Education}
  \entry*[May 2026 (expected)]%
    {\textbf{University of Missouri, Columbia, MO --- Ph.D. in Physics}
    \begin{itemize}
      \item Dissertation: \emph{Investigating Disorder in van der Waals Thin Films}
    \item Advisor: \href{https://physics.missouri.edu/people/miceli}{Dr.\ Paul Miceli}
    \end{itemize}}
  %
  \entry*[May 2022]%
    {\textbf{University of Missouri, Columbia, MO --- M.S. in Physics}}
  %
    \entry*[May 2020]{%
      \begin{minipage}[t]{\linewidth}
        \textbf{Missouri State University, Springfield, MO - B.S.in Physics}\\
         \hspace*{14em} \textit{Minor: Mathematics, Chemistry}
      \end{minipage}}

\end{rubric}


% Professional and research work history
\begin{rubric}{Research Experience}

    \entry*[2021--Present]%
    \textbf{Graduate Researcher}, University of Missouri; Advisor: Dr. Paul Miceli
    \begin{itemize}
        \item Developed \textbf{Python}-based quantitative Grazing Incident Wide Angle Xray Scattering (GIWAXS) \textbf{simulations} with \textbf{Reverse Monte Carlo} methods to extract occupancies, anisotropic Debye--Waller factors, mosaicity, and experimental geometry.
        \item Extended GIWAXS to \textbf{model diffuse scattering from stacking faults} to quantify defect densities.
        \item Grew controlled-phase PbI$_2$ thin films via \textbf{Chemical Vapor Deposition} (CVD); validated simulated polytype fractions experimentally.
        \item Implemented \textbf{CNNs} trained on simulated GIWAXS data using \textbf{PyTorch} for automated structural analysis of van der Waals thin films.
    \end{itemize}

    \entry*[2019--2020]%
    \textbf{Research Intern}, NASA Space Consortium
    \begin{itemize}
        \item Synthesized graphene films via \textbf{pulsed laser deposition} (PLD) and \textbf{pulsed vapor deposition} (PVD); characterized via \textbf{Raman spectroscopy} and \textbf{electron microscopy}.
    \end{itemize}

    \entry*[2017--2020]%
    \textbf{Research Assistant}, Missouri State University; Advisors: Dr. Kartik Ghosh, Dr. Saibal Mitra
    \begin{itemize}
        \item Designed, built, and operated PLD system; characterized thin films via XRD, Raman, SEM/EDS, and profilometry.
    \end{itemize}

    \entry*[2019]%
    \textbf{R\&D Intern}, Dynatek Labs
    \begin{itemize}
        \item Developed software for biomedical testing and automated hardware systems.
    \end{itemize}

\end{rubric}

%!TEX encoding = UTF8
%!TEX root =../cv-llt.tex

\begin{rubric}{Technical Skills}
    \begin{tabular}{@{}p{4.5cm} p{10.5cm}@{}}
        \textbf{Programming} & \textbf{Python} (7 years), Fortran, C++, R, Git, MPI, Bash scripting, LaTeX, SQL, \textbf{Excel},\textbf{ Visual Basic Advanced} \\
        \textbf{Data Analysis} & Monte Carlo methods, Machine Learning (\textbf{PyTorch}, TensorFlow), \textbf{NumPy}, pandas, SciPy \\
        \textbf{Data Visualization} & \textbf{Matplotlib}, \textbf{Plotly}, OriginLab, MATLAB, Jupyter Notebooks, Dash \\
        \textbf{Instrumentation} & \textbf{X-ray/neutron scattering}, Chemical Vapor Deposition (CVD), Pulsed Laser Deposition (PLD), Scanning Electron Microscopy (SEM), Raman spectroscopy \\
        \textbf{Communication} & \textbf{Technical writing}, Video and Animation creation/editing, Grant proposal development, Peer-review process 
    \end{tabular}
\end{rubric}

% Bibliography output for published work
\makerubrichead{Research Publications}

%% Assuming you've already given \addbibresource{own-bib.bib} in the main doc. Right? Right???
\nocite{*}

\printbibliography[heading=subbibliography,title={Journal Articles},type=article]

\printbibliography[heading=subbibliography,title={In Review / In Preparation},keyword=inprep]


\printbibliography[heading=subbibliography,title={Conference Proceedings},type=inproceedings]

\pagebreak
\begin{rubric}{Teaching Experience}

  \entry*[2018--2023]%
    \textbf{Instructor and Teaching Assistant}, University of Missouri and Missouri State University \RMPbadge\par
    \begin{itemize}
      \item Calculus-based Mechanics, Electricity \& Magnetism, and Introductory C++ Programming
    \end{itemize}

  \entry*[2021--Present]%
    \textbf{Academic Tutor}, Physics Courses, University of Missouri, Columbia, MO

  \entry*[2018--2021]%
    \textbf{ACT Prep Tutor}, Club Z!, Springfield, MO

  \entry*[2014--2020]%
    \textbf{Martial Arts Coach}, Dunham’s Martial Arts, Springfield, MO
\end{rubric}


% Professional memberships and community activities
\begin{rubric}{Leadership, Service \& Outreach}

\subrubric{Outreach and Service}
\entry*[2023--2024] \textbf{Vice-President, Physics and Astronomy Graduate Student Association (PAGSA)}, University of Missouri.
\entry*[2022--2024] \textbf{Director, PAGSA Mental Health Wellness Program}, University of Missouri.
\entry*[2022--2024] \textbf{Research Outreach}, University of Missouri.
\entry*[2022--2023] \textbf{President, Physics and Astronomy Graduate Student Association (PAGSA)}, University of Missouri.
\entry*[2022--2023] \textbf{Coalition of Graduate Workers Diversity Officer}, University of Missouri.
\entry*[2022]       \textbf{$\Sigma \Pi \Sigma$ Physics Congress – Presentation/Poster Judge}, Washington, DC.
\entry*[2018--2019] \textbf{College of Natural and Applied Sciences Leadership Board}, Missouri State University.
\entry*[2017--2020] \textbf{SPS High School Engagement}, Missouri State University.

\subrubric{Awards}
\entry*[2023] \textbf{Outstanding Student Research Presentation}, Neutron Scattering Society.
\entry*[2023] \textbf{Excellence in Physics Fergason Scholarship}, University of Missouri.
\entry*[2023] \textbf{Green Chalk Teaching Award}, University of Missouri.
\entry*[2023] \textbf{Rangel-Boain Travel Award}, University of Missouri.
\entry*[2022] \textbf{Newell S. Gingrich Physics Scholarship}, University of Missouri.
\entry*[2022] \textbf{Excellence in Student Leadership}, Graduate Professional Council, University of Missouri.
\entry*[2022] \textbf{Excellence in Undergraduate Teaching}, University of Missouri.
\entry*[2021] \textbf{O.M. Stewart Scholarship}, University of Missouri.

\end{rubric}

\end{document}
