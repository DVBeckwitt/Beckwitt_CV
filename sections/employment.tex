% Professional and research work history
\begin{rubric}{Research Experience}

  \entry*[2021--Present]%
  \textbf{\href{https://physics.missouri.edu/news/celebrating-student-awards-scholarships-and-recognition}{Graduate Research Assistant}}, University of Missouri%
  \begin{itemize}
\item Developed \textbf{Python}-based GIWAXS forward-modeling framework to extract site occupancies, anisotropic Debye–Waller factors, mosaicity, and geometric parameters
(\href{https://www.researchgate.net/publication/377264935_Quantifying_Orientational_Order_of_PbI2_van_der_Waals_Films_with_X-ray_Diffraction_using_an_Area_Detector}{APS Mar 2023}).
    \item Extended GIWAXS to \textbf{model diffuse scattering} from stacking faults to \textbf{quantify defect densities} (\href{https://meetings.aps.org/Meeting/MAR24/Session/S64.3}{APS Mar 2024}).
    \item Grew phase‑controlled PbI$_2$ films via \textbf{Chemical Vapor Deposition}; validated polytype fractions (\href{https://doi.org/10.1021/acsami.3c14559}{ACS AMI 2023}).
    \item \textbf{Implemented CNNs} on simulated GIWAXS data using \textbf{PyTorch} for automated structure analysis.
  \end{itemize}

  \entry*[2019--2020]%
  \textbf{Research Intern}, NASA Space Consortium%
  \begin{itemize}
    \item \textbf{Synthesized graphene films} via PLD/PVD; characterized with \textbf{Raman spectroscopy} and \textbf{electron microscopy}.
  \end{itemize}

  \entry*[2017--2020]%
  \textbf{\href{https://www.mrs.org/meetings-events/annual-meetings/archive/profile/David-Beckwitt-}{Research Assistant}}, Missouri State University (Advisors: Dr. K. Ghosh, Dr. S. Mitra)%
  \begin{itemize}
    \item Designed and \textbf{built} a \textbf{PLD system}; characterized thin films by XRD, Raman, SEM/EDS, profilometry (\href{https://www.mrs.org/meetings-events/annual-meetings/archive/profile/David-Beckwitt-}{MRS Spring 2019}).
  \end{itemize}

  \entry*[2019]%
  \textbf{\href{https://dynateklabs.com}{R\&D Intern}}, Dynatek Labs%
  \begin{itemize}
    \item \textbf{Developed software} for biomedical testing and automated hardware systems.
  \end{itemize}

\end{rubric}
