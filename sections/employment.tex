% Professional and research work history
\begin{rubric}{Research Experience}

  \entry*[2021--Present]%
  \textbf{\href{https://physics.missouri.edu/news/celebrating-student-awards-scholarships-and-recognition}{Graduate Research Assistant}}, University of Missouri%
  \begin{itemize}
    \item Developed \textbf{Python}-based quantitative Grazing‑Incidence Wide‑Angle X‑ray Scattering (GIWAXS) simulations with Reverse Monte Carlo methods to extract occupancies, anisotropic Debye–Waller factors, mosaicity, and experimental geometry. (\href{https://www.researchgate.net/publication/377264935_Quantifying_Orientational_Order_of_PbI2_van_der_Waals_Films_with_X-ray_Diffraction_using_an_Area_Detector}{APS March Meeting 2023})
    \item Extended GIWAXS to model diffuse scattering from stacking faults to quantify defect densities. (\href{https://meetings.aps.org/Meeting/MAR24/Session/S64.3}{APS March Meeting 2024})
    \item Grew controlled-phase PbI$_2$ thin films via \textbf{Chemical Vapor Deposition} (CVD); validated simulated polytype fractions experimentally. (\href{https://doi.org/10.1021/acsami.3c14559}{ACS Appl. Mater. Interfaces 2023})
    \item Implemented CNNs trained on simulated GIWAXS data using PyTorch for automated structural analysis of van der Waals thin films. 
  \end{itemize}

  \entry*[2019--2020]%
  \textbf{Research Intern}, NASA Space Consortium%
  \begin{itemize}
    \item Synthesized graphene films via pulsed laser deposition (PLD) and pulsed vapor deposition (PVD); characterized via Raman spectroscopy and electron microscopy.
  \end{itemize}

  \entry*[2017--2020]%
  \textbf{\href{https://www.mrs.org/meetings-events/annual-meetings/archive/profile/David-Beckwitt-}{Research Assistant}}, Missouri State University; Advisors: Dr. Kartik Ghosh, Dr. Saibal Mitra%
  \begin{itemize}
    \item Designed, built, and operated a PLD system; characterized thin films via XRD, Raman, SEM/EDS, and profilometry. (\href{https://www.mrs.org/meetings-events/annual-meetings/archive/profile/David-Beckwitt-}{MRS Spring Meeting 2019})
  \end{itemize}

  \entry*[2019]%
  \textbf{\href{https://dynateklabs.com}{R\&D Intern}}, Dynatek Labs%
  \begin{itemize}
    \item Developed software for biomedical testing and automated hardware systems.
  \end{itemize}
\end{rubric}
